\documentclass[journal]{IEEEtran}
% correct bad hyphenation here
\hyphenation{op-tical net-works semi-conduc-tor}


\begin{document}
\title{Page Rank implemented with Hornet}
\author{Giulio~Mazzi (VR406936), Samuele~Germiniani(VR409637)}

% The paper headers
\markboth{}%
{}
\maketitle

% As a general rule, do not put math, special symbols or citations
% in the abstract or keywords.
\begin{abstract}
Discussion of the parallel implementation of the Page Rank algorithm 
in CUDA, using the Hornet framework
\end{abstract}

\section{Introduction}
\IEEEPARstart{T}{his} demo file is intended to serve as a ``starter file''
for IEEE journal papers produced under \LaTeX\ using
IEEEtran.cls version 1.8b and later.
% You must have at least 2 lines in the paragraph with the drop letter
% (should never be an issue)

% needed in second column of first page if using \IEEEpubid
%\IEEEpubidadjcol

\section{Page Rank Algorithm}
PageRank is one of the most known and influential algorithms for computing the
relevance of web pages, and is used by Google, the most successful search
engine on the web. The basic idea of PageRank is that the importance of a web
page depends on the pages that link to it. For instance, we create a web page i
that includes a hyperlink to web page j. If there are a lot pages also link to
j, we then consider j is important on the web. On the hand, if j only has one
in-link, however, this link is from an authoritative web page k (like
www.google.com, www.yahoo.com, or www.bing.com), we also think j is important
because k can transfer its popularity or authority to j.
In general, if a page has k out-links, it will pass on ${\frac{1}{k}}$ of its
importance to each of the pages that it links to.
Suppose that there some pages that do not have any
out-links (we call them dangling nodes), our random surfer will get stuck on
these pages, and the importance received by these pages cannot be propagated.
In the other senario, if our web graph has two disconnected components, the
random surfer that starts from one component has no way to get into the other
component. All pages in other component will receive 0
importance.  Dangling nodes and disconnected components actually are quite
common on the Internet, considering the large scale of the web. In order to
deal with these two problems, a positive constant d between 0 and 1 (typically
0.15) is introduced, which we call the teleport parameter.
A smaller, but positive percentage of the time, the surfer will dump the
current page and choose arbitrarily a different page from the web, and 
teleports” there. The teleport parameter d reflects the probability that the
surfer quits the current page and “teleports” to a new one. Since every page
can be teleported, each page has 1 probability to be chosen.
Page rank can be implemented in many different ways, the one that we use as
reference was the push-based algorithm as seen in \cite{PR}.
We report the pseudo code:
% ......
The main idea is to compute a residual for every node, based on the ingoing
edges. When the residual is greater than a fixed threshold the
page rank value related to that residual is increased, and so the residual of
the neighbourn of the node must be recomputed. The algorithm uses a queue to
store the node that must be updated.

\section{Parallel implementation} 
For the parallel algorithm we use the Hornet framework. Hornet provides
usefull and efficent primivites for graph computation on GPUs.
- descrizione pseudocodice
- relazionoe primitive <=> pseudocodice
- relazione con il codice sequenziale
- nota sull'uso della coda... 

\section{Experimental Results} 
\subsubsection{Testing Environment}
We tested our implementation on a AMD Phenom(tm) II X6 1055T with 8gb RAM and 
a GeForce GTX 780.
We compare the performances of our algorithm, a sequential implementation and
another GPU implementation that relied on the Gunrock framework.
\subsubsection{dataset}
We selected some real world graphs with different features such as
nodes number, edges number and average degree.
Must be noted that we removed all the self loop from the graph, because a self
loop in page rank doesn't have much sense, as it makes no sense, for a node,
to give influence to itself.
A complete description of the graphs can be found in table \ref{graph}.
We set 0.001 as threshold for the algorithm and 0.85 as the teleport parameter.

\begin{table}[h]
\centering
\setlength\tabcolsep{5pt}
\caption{Graph Features}
\label{graph}
\begin{tabular}{|c|r|r|c|c|}
\hline
\multicolumn{1}{|c|}{\bfseries Graph Name} & 
\multicolumn{1}{|c|}{\bfseries Nodes} & 
\multicolumn{1}{|c|}{\bfseries Edges} & 
    \multicolumn{1}{|p{0.8cm}|}{\centering \bfseries AVG\\ Degree} & 
\multicolumn{1}{|c|}{\bfseries File size} \\
\hline
belgium\_osm\_nsl.mtx*       &  1.441.295  & 3.099.940   & 2,1        & 22M       \\
cit-Patents\_nsl.mtx*        &  3.774.768  & 16.518.947  & 4,3        & 250M      \\
delaunay\_n21\_nsl.mtx*      &  2.097.152  & 12.582.816  & 6          & 90M       \\
delaunay\_n23\_nsl.mtx*      &  8.388.608  & 50.331.568  & 6          & 378M      \\
delaunay\_n24\_nsl.mtx*      &  16.777.216 & 100.663.202 & 6          & 801M      \\
great-britain\_osm\_nsl.mtx* &  7.733.822  & 16.313.034  & 2,1        & 2123M     \\
hollywood-2009\_nsl.mtx*     &  1.139.905  & 112.751.422 & 98,9       & 757M      \\
italy\_osm\_nsl.mtx*         &  6.686.493  & 14.027.956  & 2,1        & 105M      \\
kron\_g500-logn21\_nsl.mtx*  &  2.097.152  & 182.081.864 & 86,8       & 1.5G      \\
ljournal-2008\_nsl.mtx*      &  5.363.260  & 77.991.514  & 14,5       & 1.2G      \\
rgg\_n\_2\_23\_s0\_nsl.mtx*  &  8.388.608  & 127.002.786 & 15,1       & 953M      \\
roadNet-CA\_nsl.mtx*         &  1.971.281  & 5.533.214   & 2,8        & 40M       \\
road\_usa\_nsl.mtx*          &  23.947.347 & 57.708.624  & 2,4        & 470M      \\
soc-LiveJournal1\_nsl.mtx*   &  4.847.571  & 68.475.391  & 14,1       & 958M      \\
wb-edu\_nsl.mtx*             &  9.845.725  & 55.311.626  & 5,6        & 834M      \\
webbase-1M\_nsl.mtx*         &  1.000.005  & 2.105.531   & 2,1        & 52M       \\
\hline
\end{tabular}
\end{table}

\subsubsection{results}
All the experimental results are shown in table \ref{experiments}.
The results make clear that the parallel implementation can grant meaningfull
speed-up over the sequential implementation. The speed-up isn't fixed but
depends on the graph feature.
The more interesting speed-up is related to Gunrock, Hornet performs better
in nearly all the graph with a speed-up around 5x, except for some graphs in
which the permormance are similar, or show a realy high speed-up (more than
20x).
We have some theories regarding the speed-up, or lack off. First, we notice
that an high average degree result in a realy similar execution time.
This is probably related to the fact that both the algorithm use an ${atomicAdd}$
instruction to calculate the new page\_rank, so the overhead of this function
reduces the advantage of Hornet over Gunrock.
Second, we notice that when the speed up is realy high the number of iteration
of the two algorithm are strongly different. That probably due to better 
implemetation choices in our algorithm w.r.t. the Gunrock one.

\begin{table}[]
\centering
\setlength\tabcolsep{1.5pt}
\caption{Experimental results}
\label{experiments}
\begin{tabular}{|c|c|c|c|c|c|c|c|}
\hline
\multicolumn{1}{|p{3.5cm}|}{\bfseries \centering Graph Name} & 
\multicolumn{1}{|p{2cm}|}{\bfseries \centering Sequential(ms)} & 
\multicolumn{1}{|p{2cm}|}{\bfseries \centering Hornet(ms)} & 
\multicolumn{1}{|p{2cm}|}{\bfseries \centering Gunrock(ms)} & 
\multicolumn{1}{|p{2cm}|}{\bfseries \centering Hornet\\ iterations} & 
\multicolumn{1}{|p{2cm}|}{\bfseries \centering Gunrock\\ iterations} & 
\multicolumn{1}{|p{2cm}|}{\bfseries \centering Sequential\\ speedup} & 
\multicolumn{1}{|p{2.5cm}|}{\bfseries \centering Gunrock/Hornet\\ speedup} \\
\hline
belgium\_osm\_nsl.mtx*       & 1826,8         & 34,5       & 193,27      & 30                & 34                 & 52,95              & 5,60                   \\
cit-Patents\_nsl.mtx*        & 2254,2         & 93,4       & 952,69      & 6                 & 38                 & 24,13              & 10,20                  \\
delaunay\_n21\_nsl.mtx*      & 4321           & 94,7       & 544,03      & 30                & 32                 & 45,63              & 5,74                   \\
delaunay\_n23\_nsl.mtx*      & 16761,4        & 352        & 2123,37     & 30                & 33                 & 47,62              & 6,03                   \\
delaunay\_n24\_nsl.mtx*      & 33636,9        & 700,2      & 4216,64     & 30                & 33                 & 48,04              & 6,02                   \\
great-britain\_osm\_nsl.mtx* & 10052,7        & 163,3      & 902,54      & 30                & 34                 & 61,56              & 5,53                   \\
hollywood-2009\_nsl.mtx*     & 23527,3        & 649,2      & 617,56      & 30                & 39                 & 36,24              & 0,95                   \\
italy\_osm\_nsl.mtx*         & 6863,5         & 135,4      & 780,32      & 30                & 34                 & 50,69              & 5,76                   \\
kron\_g500-logn21\_nsl.mtx*  & 99557,2        & 3070,8     & 3562,91     & 28                & 33                 & 32,42              & 1,16                   \\
ljournal-2008\_nsl.mtx*      & 26369          & 564,6      & 1061,6      & 26                & 40                 & 46,70              & 1,88                   \\
rgg\_n\_2\_23\_s0\_nsl.mtx*  & 33779,3        & 894,8      & 938,46      & 30                & 33                 & 37,75              & 1,05                   \\
roadNet-CA\_nsl.mtx*         & 2407,1         & 51,3       & 332,09      & 30                & 35                 & 46,92              & 6,47                   \\
road\_usa\_nsl.mtx*          & 39578,1        & 533,3      & 3340,77     & 30                & 35                 & 74,21              & 6,26                   \\
soc-LiveJournal1\_nsl.mtx*   & 30524,5        & 717        & 1369,89     & 26                & 39                 & 42,57              & 1,91                   \\
wb-edu\_nsl.mtx*             & 9221,9         & 351,3      & 719,34      & 23                & 39                 & 26,25              & 2,05                   \\
webbase-1M\_nsl.mtx*         & 237,4          & 27,6       & 759,85      & 28                & 41                 & 8,60               & 27,53                 \\
\hline
\end{tabular}
\end{table}

\section{Conclusion}
hornet è figo ( e soprattuto più figo di gurnrock)
vantaggi di hornet
svantaggi di hornet

\begin{thebibliography}{1}

\bibitem{PR}
J. J. Whang, A. Lenharth, I. S. Dhillon, and K. Pingali, \emph{Scalable Data-driven PageRank: Algorithms, System Issues, and Lessons Learned}, University of Texas at Austin, Austin TX 78712, USA

\end{thebibliography}

\end{document}

